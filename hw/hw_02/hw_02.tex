\documentclass[12pt, a4paper, UTF8]{ctexart}
\usepackage{amsmath, amsthm, amssymb, graphicx}
\usepackage{geometry}
\geometry{scale=0.85}


\usepackage{ctex}
\usepackage{amssymb}
\usepackage{bm}
\usepackage{geometry}
\usepackage{amsmath}

\usepackage{pifont}
\newcommand{\whiteding}[1]{\ding{\numexpr191+#1\relax}}

\begin{document}
	
	\begin{center} \large \textbf{Solution to Exercises-2} \end{center}
	
	\begin{flushleft} 
		
		\textbf{Course}: Matrix Analysis and Applications\\
		
		\textbf{Instructor}: Minghua Xia\\
		
		\textbf{TA}: Qiyong Chen (E-mail: chenqy263@mail2.sysu.edu.cn)\\
		
		\ \\
		
		\textit{I hereby certify that all the work in these exercises is mine alone. I have neither received assistance from another person or group, nor have I given assistance to another person.}\\
		
		\ \\
		

		
		\noindent Name:\underline{\qquad\qquad\qquad} \quad Student ID:\underline{\qquad\qquad\qquad} \quad
		Signature:\underline{\qquad\qquad\qquad} \quad Date:\underline{\qquad\qquad\qquad} \\
		
	\end{flushleft}
	
	
	
	\begin{enumerate}
		%========================== 第1题 ========================%
		%========================== 题目 ==========================%
		\item
		Find the eigenvalues and eigenvectors of the matrices
		$$\bm{A}=
			\begin{bmatrix}
			1 & -1 \\
			1 & 1 
			\end{bmatrix}
		,
		\bm{B}=
			\begin{bmatrix}
			-3 & 1 & -3 \\
			20 & 3 & 10 \\
			2 & -2 & 4 
			\end{bmatrix}.
		$$

		%========================== 答案 ==========================%
		\textbf{Solution}:
		
		\textbf{For $\bm{A}$:}

		Firstly, find the eigenvalues of $\bm{A}$:

		\begin{equation*}
			det(\bm{A}-\lambda\bm{I})=
			\left|\begin{matrix}
			1-\lambda & -1 \\
			1 & 1-\lambda
			\end{matrix}\right|=(1-\lambda)^2+1=0,
		\end{equation*}

		therefore, we have the eigenvalues of $\bm{A}$: $\lambda_1=1-i,\lambda_2=1+i$.

		Let $\bm{x}$ be the eigenvector of $\bm{A}$, therefore, we have $(\bm{A}-\lambda\bm{I})\bm{x}=\bm{0}$, substitute $\lambda_1=1-i,\lambda_2=1+i$ into the equation.

		\begin{equation*}
			\begin{bmatrix}
				-i & 1 \\
				1 & -i
			\end{bmatrix}\bm{x} = \bm{0} \Longrightarrow \bm{x} = k \begin{bmatrix}
				i \\ 1
			\end{bmatrix}, k \in \mathbb{R}.
		\end{equation*}

		\begin{equation*}
			\begin{bmatrix}
				i & -1 \\ 1 & i
			\end{bmatrix}\bm{x} = \bm{0} \Longrightarrow \bm{x} = k \begin{bmatrix}
				1 \\ i
			\end{bmatrix}, k \in \mathbb{R}.
		\end{equation*}
		
		In summary, the eigenvalues of $\bm{A}$ are $\lambda_1=1-i,\lambda_2=1+i$, and the eigenvectors of $\bm{A}$ are 
		$ \bm{x}_{1}=\begin{bmatrix}
			i\\
			1
		\end{bmatrix}$ for $\lambda_1$, and 
		$\bm{x}_{2}=\begin{bmatrix}
			1\\
			i
		\end{bmatrix}$ for $\lambda_2$.

		\textbf{For $\bm{B}$:}

		Firstly, find the eigenvalues of $\bm{B}$:

		\begin{equation*}
			det(\bm{B}-\lambda\bm{I})=
			\begin{bmatrix}
			-3-\lambda	& 1 & -3\\
			20	&3 -\lambda & 10\\
			2	& -2 &4-\lambda
			\end{bmatrix}=-\lambda^{3}+4\lambda^{2}+3\lambda-18=-(\lambda+2)(\lambda-3)^{2}=0, 
		\end{equation*}

		therefore, $\lambda_{1}=-2$, $\lambda_{2}=\lambda_{3}=3$.

		Let $\bm{x}$ be the eigenvector of $\bm{B}$, therefore, we have $(\bm{B}-\lambda\bm{I})\bm{x}=\bm{0}$, substitute $\lambda_{1}=-2$, $\lambda_{2}=\lambda_{3}=3$ into the equation.

		\begin{equation*}
			\begin{bmatrix}
				-1	& 1  & -3\\
				20	& 5  & 10\\
				2	& -2 & 6
			\end{bmatrix}\bm{x} = \bm{0} \Longrightarrow \begin{bmatrix}
				1 & 0 & 1\\
				0 & 1 & -2\\
				0 & 0 & 0
			\end{bmatrix}\bm{x} = \bm{0} \Longrightarrow \bm{x} = k \begin{bmatrix}
				-1 \\ 2 \\ 1
			\end{bmatrix}, k \in \mathbb{R}.
		\end{equation*}

		\begin{equation*}
			\begin{bmatrix}
				-6	& 1  & -3\\
				20	& 0  & 10\\
				2	& -2 & 1
			\end{bmatrix}\bm{x} = \bm{0} \Longrightarrow \begin{bmatrix}
				1 & 0 & \frac{1}{2}\\
				0 & 1 & 0\\
				0 & 0 & 0
			\end{bmatrix}\bm{x} = \bm{0} \Longrightarrow \bm{x} = k \begin{bmatrix}
				-\frac{1}{2} \\ 0 \\ 1
			\end{bmatrix}, k \in \mathbb{R}.
		\end{equation*}

		In summary, the eigenvalues of $\bm{B}$ are $\lambda_{1}=-2$, $\lambda_{2}=\lambda_{3}=3$, and the eigenvectors of $\bm{B}$ are 
		$ \bm{x} = \begin{bmatrix}
			-1 \\ 2 \\ 1
		\end{bmatrix}$ for $\lambda_1$, and 
		$\bm{x} = \begin{bmatrix}
			-\frac{1}{2} \\ 0 \\ 1
		\end{bmatrix}$ for $\lambda_2$ and $\lambda_3$.

		%========================== 第2题 ========================%
		%========================== 题目 ==========================%
		\item
		Determine if either of the following matrices is diagonalizable
		$$\bm{A}=
			\begin{bmatrix}
			-1 & -1 & -2 \\
			8 & -11 & -8 \\
			-10 & 11 & 7 
			\end{bmatrix}
		,
		\bm{B}=
			\begin{bmatrix}
			1 & -4 & -4 \\
			8 & -11 & -8 \\
			-8 & 8 & 5 
			\end{bmatrix}.
		$$

		%========================== 答案 ==========================%
		\textbf{Solution:}
		
		\textbf{For $\bm{A}$}:
		\begin{equation*}
			det(\bm{A}-\lambda \bm{I})= 
			\left|\begin{matrix} 
				-1-\lambda & -1 & -2 \\ 
				8 & -11-\lambda  & -8 \\ 
				-10 & 11 & 7-\lambda 
			\end{matrix} \right| =-\lambda^3-5\lambda^2-3\lambda+9=-(\lambda +3)^2(\lambda -1)=0
		\end{equation*}.

		Therefore, the eigenvalues of $\bm{A}$ are $\lambda_1=\lambda_2=-3, \lambda_3=1$.

		For $\lambda_1=\lambda_2=-3$, we can easily calculate to get $rank(\bm{A}-\lambda_1\bm{I})$ = 2. Therefore, $geo \; mult_{\bm{A}}(\lambda_1) = \mathcal{N}(\bm{A}-\lambda_1\bm{I}) = 3 - rank(\bm{A}-\lambda_1\bm{I}) = 1 \neq 2 = alg \; mult_{\bm{A}}(\lambda_1)$, $\lambda_1=\lambda_2=-3$ are not semisimple.

		Because $\lambda_1=\lambda_2=-3$ are not semisimple, $\bm{A}$ is not diagonalizable.

		\textbf{For $\bm{B}$}:
		\begin{equation*}
			det(\bm{B}-\lambda \bm{I})= 
			\left| \begin{matrix} 
				1-\lambda & -4 & -4 \\ 
				8 & -11-\lambda & -8 \\ 
				-8 & 8 & 5-\lambda 
			\end{matrix} \right|=-\lambda^3-5\lambda^2-3\lambda+9=-(\lambda+3)^2(\lambda-1)=0
		\end{equation*}.

		Therefore, the eigenvalues of $\bm{B}$ are $\lambda_1=\lambda_2=-3, \lambda_3=1$.

		For $\lambda_1=\lambda_2=-3$, we can easily calculate to get $rank(\bm{B}-\lambda_1\bm{I})$ = 1. Therefore, $geo \; mult_{\bm{B}}(\lambda_1) = \mathcal{N}(\bm{B}-\lambda_1\bm{I}) = 3 - rank(\bm{B}-\lambda_1\bm{I}) = 2 = alg \; mult_{\bm{A}}(\lambda_1)$. So $\lambda_1=\lambda_2=-3$ are semisimple.

		In the same way, we can know that $\lambda_3=1$ is also semisimple.

		In summary, $\lambda_1=\lambda_2=-3, \lambda_3=1$ are semisimple, so $\bm{B}$ is diagonalizable.

		%========================== 第3题 ========================%
		%========================== 题目 ==========================%
		\item
		Estimate the eigenvalues of 
		$\bm{A}=
			\begin{bmatrix}
			5 & 1 & 1\\
			0 & 6 & 1\\
			1 & 0 & -5  
			\end{bmatrix}.
		$

		%========================== 答案 ==========================%
		\textbf{Solution}:

		We can use Gerschgorin circles to estimate the eigenvalues. 
		
		Firstly we use row vectors of $\bm{A}$, then we get three circles: $C_{1}(5,2),C_{2}(6,1),C_{3}(-5,1)$. 
		
		In the same way, we use column vectors of $\bm{A}$ and get three circles: $C_{1}(5,1),C_{2}(6,1),C_{3}(-5,2)$.

		We need to choose circles which have shorter radius to estimate eigenvalues. In the reason, the ciecles used to estimate eigenvalues are : $C_{1}(5,1),C_{2}(6,1),C_{3}(-5,1)$.

		%========================== 第4题 ========================%
		%========================== 题目 ==========================%
		\item
		Determine the induced norm $\|\bm{A}\|_{2}$ as well as $\|\bm{A}^{-1}\|_{2}$ for the nonsingular matrix
		$$
		\bm{A}=\frac{1}{\sqrt{3}}
			\begin{bmatrix}
			3 & -1\\
			0 & \sqrt{8}
			\end{bmatrix}.
		$$

		%========================== 答案 ==========================%

		\textbf{Solution}:

		We know that $||\bm{A}||_2 = \sqrt{\lambda_{max} (\bm{A}^H \bm{A})}$

		For $||\bm{A}||_2$, 
		\begin{equation*}
			\bm{A}^H\bm{A}=\frac{1}{\sqrt{3}}
			\begin{bmatrix}
				3 & 0\\ 
				-1 & \sqrt{8} 
			\end{bmatrix} \frac{1}{\sqrt{3}}
			\begin{bmatrix} 
				3 & -1\\ 
				0&\sqrt{8} 
			\end{bmatrix}=
			\begin{bmatrix}
				3&-1\\ 
				-1&3 
			\end{bmatrix}.
		\end{equation*}

		Therefore,
		\begin{equation*}
			det(\bm{A}^H\bm{A}-\lambda\bm{I})= 
			\left|\begin{matrix}
				3-\lambda & -1 \\ 
				-1 & 3-\lambda 
			\end{matrix} \right|
			=\lambda ^2 -6 \lambda  +8=(\lambda -4)(\lambda -2)=0.
		\end{equation*}

		$\lambda =2, 4$, so $||\bm{A}||_2 = \sqrt{\lambda_{max}(\bm{A}^H \bm{A})}=\sqrt{4}=2$.

		Before calculating $||\bm{A}^{-1}||_2$, we try to proof $\bm{A}^H \bm{A}$ and $\bm{A} \bm{A}^H$ have same eigenvalues.

		Ler $\lambda$ be the eigenvalue of $\bm{A}^H\bm{A}$, and $\mathbf{x}$ is the corresponding eigenvector, then
		\begin{equation*}
			\bm{A}^H \bm{A} \bm{x} = \lambda \bm{x}.
		\end{equation*}

		Let $\bm{y} = \bm{A}\bm{x}$, then
		\begin{equation*}
			\bm{A}\bm{A}^H\bm{y} = \bm{A}\bm{A}^H\bm{y}\bm{A}\bm{x} = \bm{A}\lambda \bm{x} = \lambda\bm{y},
		\end{equation*}
		therefore, $\lambda$ is also the eigenvalue of $\bm{A}\bm{A}^H$ and $\bm{y}$ is the corresponding eigenvector.

		Above all, we know that $\bm{A}^H \bm{A}$ and $\bm{A} \bm{A}^H$ have same eigenvalues. When the eigenvalues of $\bm{A}^H \bm{A}$ are $\lambda =2, 4$, the eigenvalues of $\bm{A}\bm{A}^H$ are also $\lambda =2, 4$. Furthermore, the eigenvalues of $(\bm{A}\bm{A}^H)^{-1}$ are $\lambda^{-1} =\frac{1}{2}, \frac{1}{4}$.

		For $||\bm{A}^{-1}||_2$, 
		\begin{align*}
			||\bm{A}^{-1}||_2 &= \sqrt{\lambda_{max}((\bm{A}^{-1})^{H}\bm{A}^{-1})} \\
			&= \sqrt{\lambda_{max}((\bm{A}^{H})^{-1}\bm{A}^{-1})} \\
			&= \sqrt{\lambda_{max}((\bm{A}^{H}\bm{A})^{-1})} \\
			&= \sqrt{\frac{1}{2}} \\
			&=\frac{\sqrt{2}}{2}
		\end{align*}
		
		%========================== 第5题 ========================%
		%========================== 题目 ==========================%
		\item
		Let $\bm{A}\in\mathbb{C}^{m\times n}$. Please prove the following statements.
		\begin{enumerate}
			\item $\| \bm{A}\|_{p}=\max\limits_{\| \bm{x}\|_{p}=1}\| \bm{Ax}\|_{p}$ is a norm for any $p\geq 1$;
			\item $\| \bm{Ax}\|_{p}\leq\| \bm{A}\|_{p}\| \bm{x}\|_{p}$ holds for any $p\geq 1$;
			\item $\| \bm{AB}\|_{p}\leq\| \bm{A}\|_{p}\| \bm{B}\|_{p}$ holds for any $p\geq 1$;
			\item $\| \bm{QAU}\|_{F}=\| \bm{A}\|_{F}$ holds for any unitary matrices $\bm{Q}\in \mathbb{C}^{m\times m}$ and $\bm{U}\in \mathbb{C}^{n\times n}$;
			\item $\| \bm{QAU}\|_{2}=\| \bm{A}\|_{2}$ holds for any unitary matrices $\bm{Q}\in \mathbb{C}^{m\times m}$ and $\bm{U}\in \mathbb{C}^{n\times n}.$
		\end{enumerate}

		%========================== 答案 ==========================%
		%========================== (a) ==========================%
		\textbf{(a)Proof}:
		
		\text{1.Non-negetivity}:

		Obviously: $||\bm{Ax}||_p \geq 0$ for any $p \geq 1$, therefore, $\|\bm{A}\|_{p}=\max\limits_{\|\bm{x}\|_{p}=1}\|\bm{Ax}\|_{p} \geq 0$ for any $p\geq 1$.
		
		\text{2.Separate points}:

		Let $\| \bm{A}\|_{p}=\max\limits_{\| \bm{x}\|_{p}=1}\| \bm{Ax}\|_{p} = 0$, now we have $||\bm{Ax}||_p=0$ for any $\bm{x} \in \mathbb{R}, ||\bm{x}||_p=1$. Therefore, only $\bm{A} = \bm{0}$ meets the condition. 
		
		
		\text{3.Triangle inequality}:

		\begin{align*}
			||\bm{A}||_p+||\bm{B}||_p &=\max \limits_{\| \bm{x}\|_{p}=1}\| \bm{Ax}\|_{p}+\max \limits_{\| \bm{x}\|_{p}=1}\| \bm{Bx}\|_{p} \\
			&\geq\max\limits_{\|\bm{x}\|_{p}=1}(\|\bm{Ax}\|_{p}+\|\bm{Bx}\|_{p}) \\ 
			&\geq\max\limits_{\|\bm{x}\|_{p}=1}(\|\bm{Ax} + \bm{Bx}\|_{p}) \\
			&\geq \max\limits_{\| \bm{x}\|_{p}=1}(\| (\bm{A}+\bm{B})\bm{x}\|_{p})=||\bm{A}+\bm{B}||_p
		\end{align*}
		
		Therefore, we can proof that  $\displaystyle \| \bm{A}\|_{p}=\max_{\| \bm{x}\|_{p}=1}\| \bm{Ax}\|_{p}$ is a norm for any $p\geq 1$.
		
		
		\text{4.Absolute homogeneity}:

		$||\alpha\bm{A}||=\max\limits_{\|\bm{x}\|_{p}=1}\|\alpha\bm{Ax}\|_{p}=|\alpha|\max\limits_{\|\bm{x}\|_{p}=1}\|\bm{Ax}\|_{p}=|\alpha|||\bm{A}||_p$

		%========================== (b) ==========================%
		\textbf{(b)Proof}:

		We know that $||\bm{A}||_p\triangleq\mathop{sup}\limits_{\bm{x} \neq 0}\frac{||\bm{Ax}||_p}{||\bm{x}||_p} \geq\frac{||\bm{Ax}||_p}{||\bm{x}||_p}$, for $\bm{x}\neq\bm{0}$.

		Therefore, $||\bm{Ax}||_p \leq ||\bm{A}||_p ||\bm{x}||_p$ for any $p \geq 1$.
		
		%========================== (c) ==========================%
		\textbf{(c)Proof}:
		\begin{align*}
			||\bm{AB}||_p &= \mathop {max} \limits_{\|\bm{x}\|_p = 1} ||\bm{ABx}||_p \\
			&= \mathop {max} \limits_{\|\bm{x}\|_p = 1} ||\bm{A(Bx)}||_p \\
			&\leqslant \mathop {max} \limits_{\|\bm{x}\|_p = 1} ||\bm{A}||_p||\bm{Bx}||_p \\
			&=||\bm{A}||_p||\bm{B}||_p
		\end{align*}
		
		Therefore, $\|\bm{AB}\|_{p}\leq\|\bm{A}\|_{p}\|\bm{B}\|_{p}$ holds for any $p\geq 1$.

		%========================== (d) ==========================%
		\textbf{(d)Proof}:
		We know $\|\bm{A}\|_{F}^{2}={tr}(\bm{A}^{H}\bm{A})^{\frac{1}{2}}$, then 
		\begin{align*}
			\|\bm{Q}\bm{A}\bm{U}\|_{F}^{2}&={tr}(\bm{U}^{H}\bm{A}^{H}\bm{Q}^{H}\bm{Q}\bm{A}\bm{U})^{\frac{1}{2}} \\
			&={tr}(\bm{U}^{H}\bm{A}^{H}\bm{A}\bm{U})^{\frac{1}{2}} \\
			&={tr}(\bm{A}^{H}\bm{A})^{\frac{1}{2}} \\
			&=\|\bm{A}\|_{F}^{2}
		\end{align*}.

		%========================== (e) ==========================%
		\textbf{(e)Proof}:
		
		We know that: $||\bm{QAU}||_2=\sqrt{\lambda_{max}((\bm{QAU})^H(\bm{QAU}))}$, and $||\bm{A}||_2 =\sqrt{\lambda_{max}(\bm{A}^H\bm{A})}$. And $(\bm{QAU})^H(\bm{QAU}) = \bm{U}^H\bm{A}^H\bm{Q}^H\bm{QAU} = \bm{U}^H \bm{A}^H\bm{A}\bm{U}$ is similar to $\bm{A}^H\bm{A}$. 
		
		Therefore, $(\bm{QAU})^H(\bm{QAU})$ has the same eigenvalues with $\bm{A}^H\bm{A}$. 
		
		In the reason, $\lambda_{max}((\bm{QAU})^H(\bm{QAU}))=\lambda_{max}(\bm{A}^H\bm{A})$.

		Above all, $\|\bm{QAU}\|_{2}=\|\bm{A}\|_{2}$ holds for any unitary matrices $\bm{Q}\in \mathbb{C}^{m\times m}$ and $\bm{U}\in \mathbb{C}^{n\times n}.$

		%========================== 第6题 ========================%
		%========================== 题目 ==========================%
		\item
		Using the induced matrix norm, prove that if $A$ is nonsingular, then
		$$\min_{||\bm{x}\|=1}\|\bm{Ax}\|=\frac{1}{\|\bm{A}^{-1}\|}.$$

		%========================== 答案 ==========================%
		\textbf{Proof}:
		
		\begin{align*}
			||\bm{A}^{-1}||&=\mathop{sup}\limits_{\bm{x}\neq\bm{0}}\frac{||\bm{A}^{-1}\bm{x}||}{||\bm{x}||} \\
			&=\mathop{sup}\limits_{\bm{y}\neq\bm{0}}\frac{||\bm{y}||}{||\bm{Ay}||}, \bm{y}=\bm{A}^{-1}\bm{x} \\
			&=\mathop{sup}\limits_{\bm{y}\neq\bm{0}}\frac{1}{\frac{1}{||y||}||\bm{Ay}||} \\
			&=\mathop{sup}\limits_{\bm{y}\neq\bm{0}}\frac{1}{||\bm{A}\frac{\bm{y}}{||\bm{y}||}||} \\
			&=\mathop{max}\limits_{||\bm{z}||= 1}\frac{1}{||\bm{Az}||}, \bm{z}=\frac{\bm{y}}{||\bm{y}||}\\
			&=\frac{1}{\mathop {min}\limits_{||\bm{z}||=1}||\bm{Az}||}
		\end{align*}

		Therefore, $\mathop{min}\limits_{||\bm{x}\|=1}\|\bm{Ax}\|=\frac{1}{\|\bm{A}^{-1}\|}.$
		
		%========================== 第7题 ========================%
		%========================== 题目 ==========================%
		\item
		Let $\bm{A}\in\mathbb{C}^{m\times n}$ is a Vandemonde matrix with distinct roots. 
		Please verify that any collection of $r$ columns of $\bm{A}$, with $r\leq m$, is linearly independent.
		\\\textbf{Proof}:
		We know that: $\bm{A}= 
		\left[\begin{matrix} 
			1 & 1 & \dots & 1 \\ 
			a_1 & a_2 & \cdots & a_n \\ 
			\vdots & \vdots & \ddots & \vdots \\ 
			a_1^{m-1} & a_2^{m-1} & \cdots & a_n^{m-1}
		\end{matrix} \right]$

		%========================== 答案 ==========================%
		When $m \leq n$, any collection of $m$ columns of $\bm{A}$ can form a square matrix $\bm{A}_1$, and $\bm{A}_1$ is still a Vandemonde matrix.
		
		$\mathop{det}(\bm{A}_1)=\prod\limits_{1\leq i\leq j\leq m}(a_j-a_i) \neq 0, \quad (a_j \neq a_i)  \Longrightarrow \mathop{rank}(\bm{A}_1)=m$
		
		Therefore, any collection of $r$ columns of $\bm{A}$, with $r\leq m$, is linearly independent.
		
		When $m < n$, any collection of $n$ rows of $\bm{A}$ can form a square matrix $\bm{A}_2$, and $\bm{A}_2$ is still a Vandemonde matrix.

		$\mathop{det}(\bm{A}_2)=\prod\limits_{1\leq i\leq j\leq n}(a_j-a_i)\neq 0, \quad (a_j\neq a_i) \Longrightarrow \mathop{rank}(\bm{A}_1)=n$
		
		Therefore, any collection of $r$ columns of $\bm{A}$ with $r\leq n < m$ is linearly independent.

		%========================== 第8题 ========================%
		%========================== 题目 ==========================%
		\item%第8题
		Suppose $\bm{A}$ has eigenvalues $0,3,5$ with independent eigenvectors $\bm{u}, \bm{v}, \bm{w}$.
		\begin{enumerate}
			\item Give a basis for the nullspace and a basis for the column space;
			\item Find a particular solution to $\bm{Ax}=\bm{v}+\bm{w}$;
			\item Find all solutions to $\bm{Ax}=\bm{v}+\bm{w}$;
			\item $\bm{Ax}=\bm{u}$ has no solution. If it did then \underline{\makebox[3em]{}} would be in the column space.
		\end{enumerate}

		%========================== 答案 ==========================%
		\textbf{Solution}:

		\textbf{(a)}
		$\bm{A}$ has two non-zero eigenvalues, so $rank(\bm{A}) = 2$. Furthermore, $rank(\mathcal{N}(\bm{A})) = 3 - rank(\bm{A}) = 1, rank(\mathcal{C}(\bm{A})) = rank(\bm{A}) = 2$. $\bm{u}$ is the eigenvector of $\bm{A}$ corresponding $\lambda = 0$, therefore, $\bm{A} \bm{u} = \bm{0}$, $\bm{u}$ is a basis for the nullspace.
		
		Obviously, $\bm{v}$ and $\bm{w}$ give a basis for the column space.

		\textbf{(b)}
		We have 
		\begin{equation*}
			\bm{A}\bm{v} = 3\bm{v} \Longrightarrow \bm{v} = \frac{1}{3}\bm{A}\bm{v},
		\end{equation*}
		\begin{equation*}
			\bm{A}\bm{w} = 5\bm{w} \Longrightarrow \bm{u} = \frac{1}{5}\bm{A}\bm{u}.
		\end{equation*}
		Therefore,
		\begin{equation*}
			\bm{A}(\frac{1}{3}\bm{v}+\frac{1}{5}\bm{w}) = \bm{v} + \bm{w}
		\end{equation*}
		Above all, $\frac{1}{3}\bm{v}+\frac{1}{5}\bm{w}$ is a particular solution to $\bm{Ax}=\bm{v}+\bm{w}$.

		\textbf{(c)}
		Because $\bm{A} k \bm{u} = \bm{0}, k \in \mathbb{R}$, then $\bm{A}(\frac{1}{3}\bm{v}+\frac{1}{5}\bm{w}+k\bm{u}) = \bm{v} + \bm{w}, k \in \mathbb{R}$.
		Therefore, $\frac{1}{3}\bm{v}+\frac{1}{5}\bm{w}+k\bm{u}, k \in \mathbb{R}$ are all solutions to $\bm{Ax}=\bm{v}+\bm{w}$.
		
		\textbf{(d)}
		$\bm{Ax}=\bm{u}$ has no solution. If it did then $\bm{u}$ would be in the column space. $\bm{A}$ will have three demension contrary to the meaning of the question.
		
	\end{enumerate}
	
	
	
	
\end{document}